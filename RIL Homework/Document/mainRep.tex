\documentclass{report}

\usepackage{ptext}
\usepackage{lipsum}
\input{Boostan-UserManual}
\usepackage{graphicx}
\usepackage{tabularx}

\newword{Abstraction}{Abstraction}
{انتزاع}{}

\newword{Abstract}{Abstract}
{انتزاعی}{}

\newword{AbsoluteMinimum}{Absolute Minimum}
{کمینه مطلق}{}


\newword{AcceptableCell}{Acceptable Cell}
{سلول پذیرفتنی}{سلول‌های پذیرفتنی}

\newword{AccessBurst}{Access Burst}
{توده دسترسی}{توده‌های دسترسی}


%%% S
\newword{Sample}{Sample}
{نمونه}{نمونه‌ها}

\newword{SamplePath}{Sample Path}
{نمونه مسیر}{}

\newword{SampleSpace}{Sample Space}
{فضای نمونه}{فضای نمونه‌ها}
\newacronym{ACK}{ACK}{Acknowledgement}

\newacronym{ACI}{ACI}{Application Control Interface}

\newacronym{ACIR}{ACIR}{Adjacent Channel Interference Ratio}

\newacronym{ACLC}{ACLC}{Adaptive Configuration of Logical Channels}

\newacronym{ACLP}{ACLP}{Adjacent Channel Leakage Power}

\title{تمرین فصل دوم: معماری شبکه های تلفن همراه
}
\type{
 درس آشنایی با شبکه های تلفن همراه }
\author{غزل عربعلی - 97521396، بهاره کاوسی نژاد - 99431217}
\logofile{Pic/IUST}


\begin{document}
\pagenumbering{gobble}
\maketitle
\pagenumbering{arabic}
\chapter*{تمرین 
\lr{RIL}
}

\section*{سوال اول}
در مورد لایه بندی
 \lr{RIL}
  که در اسلایدهای پیش صحبت شد، تحقیق کنید؟

\subsection*{پاسخ}
لایه رابط رادیویی (
\lr{Radio Interface Layer}
)، به طور خلاصه
 \lr{RIL}
 ، جزء مرکزی پلتفرم اندروید است که ارتباطات 
\lr{cellular}
 را مدیریت می کند. لایه رابط رادیویی یک رابط برای مودم 
 \lr{cellular}
  فراهم می کند و با شبکه تلفن همراه برای ارائه خدمات تلفن همراه کار می کند.
   \lr{RIL}
    طوری طراحی شده است که مستقل از تراشه های مودم 
    \lr{cellular}
     عمل کند. در نهایت 
     \lr{RIL}
      مسئول مواردی مانند تماس های صوتی، پیام های متنی و اینترنت موبایل است. بدون 
      \lr{RIL}
      ، یک دستگاه
       \lr{Android}
        نمی تواند با یک شبکه تلفن همراه ارتباط برقرار کند. 
        \lr{RIL}
         تا حدی چیزی است که یک دستگاه اندرویدی را به یک گوشی هوشمند تبدیل می کند. امروزه، ارتباطات سلولی دیگر محدود به تلفن‌های همراه و تلفن‌های هوشمند نیست، زیرا تبلت‌ها و کتابخوان‌های الکترونیکی (
         \lr{eBook readers}
         )با اینترنت همراه داخلی و همیشه روشن عرضه می‌شوند. اینترنت موبایل به عهده 
         \lr{RIL}
          است و بنابراین
           \lr{RIL}
            در اکثر دستگاه های اندرویدی وجود دارد.
            در اندروید ورژن 7،
            \lr{RIL}
            بازنویسی (
            \lr{refactore}
            )
            شده است. جزئیات آن در  
            \href{https://source.android.com/docs/core/connect/ril} {این لینک}
            قابل مشاهده است.
            
            \lr{RIL}
             یک جزء کلیدی از سیستم عامل ویندوز موبایل مایکروسافت نیز است.
              \lr{RIL}
               برنامه‌های صوتی یا داده‌های بی‌سیم را قادر می‌سازد تا با مودم
                \lr{GSM/GPRS}
                 یا
                  \lr{CDMA2000 1X} 
                  در یک دستگاه
                   \lr{Windows Mobile}
                    ارتباط برقرار کنند.
                     \lr{RIL}
                     رابط سیستم را بین لایه
                      \lr{CellCore}
                       در سیستم عامل 
                       \lr{Windows Mobile}
                        و پشته پروتکل رادیویی (
                        \lr{radio protocol stack}
                        ) مورد استفاده توسط سخت افزار مودم بی سیم فراهم می کند. بنابراین 
                        \lr{RIL}
                         همچنین به
                          \lr{OEM}
                           ها اجازه می دهد تا با ارائه این رابط، انواع مودم ها را در تجهیزات خود ادغام کنند.
            
            RIL
             از دو جزء مجزا تشکیل شده است: 
             \begin{itemize}
             	\item  یک درایور 
             	\lr{RIL}
             	 که 
             	 \lr{AT command}
             	 ها و رویداد ها را پردازش می کند
             	\item یک پراکسی 
             	RIL
             	، که درخواست‌ها را از چندین مشتری به درایور 
             	\lr{RIL}
             	 واحد مدیریت می‌کند. 
             \end{itemize}
             به جز اتصالات 
             \lr{PPP}
             ، تمام تعاملات بین سیستم عامل 
             \lr{Windows Mobile }
             و 
             پشته رادیویی (
             \lr{Radio Track}
             )
              دستگاه از طریق 
              \lr{RIL}
               است. (اتصالات
                \lr{PPP}
                 ابتدا از
                  \lr{RIL}
                   برای برقراری اتصال استفاده می کنند، اما سپس 
                   \lr{RIL}
                    را دور می زنند تا مستقیماً به پورت سریال مجازی اختصاص داده شده به مودم متصل شوند.) در اصل،
                     \lr{RIL}
                      تمام درخواست های سرویس مستقیم (
                      \lr{direct service request}
                      ) از لایه های بالایی (همان 
                      \lr{TAPI}
                      ) را می پذیرد و تمام این دستورات را به دستورات پشتیبانی شده و قابل فهم برای مودم تبدیل می کند.
            
            توجه داشته باشید که
             \lr{RIL}
              مستقیماً با مودم ارتباط برقرار نمی کند. در عوض، پیوند نهایی به مودم معمولاً درایور سریال استاندارد ارائه شده توسط پلتفرم
               \lr{OEM}
                است.
                
   \subsubsection*{توضیحات بیشتر در مورد 
   \lr{RIL}
   }
   \lr{RIL}
    یک لایه واسط به نام 
    \lr{Radio Interface Layer}
     محسوب می شود 
     که واسط میان
      \lr{baseband}
       یا همان چیپ ست، 
       \lr{MT}
       یا همان
       \lr{Mobile terminal} 
        و
         \lr{Android Telephony}
          ‌است.  
   چرا که 
   \lr{MT} 
   در حالت عادی بدون استفاده از
    \lr{Android Telephony} 
    اطلاعات را در اختیار ما قرار نمی دهد.
    
   \lr{RIL}
     سه بخش دارد:
     \begin{enumerate}
     	\item
 \textbf{بخش یک
 \lr{RILJ}
 :
 }
   یک ماژول سطح جاوایی است که
    Android Telephony
     از آن استفاده می کند.
     \item
\textbf{بخش دو
\lr{Vendor RIL}
:
}
این بخش با مشارکت 
 \lr{Vendor}
   ها تولید می شود و وظیفه آن ترجمه زبان
    \lr{Android Telephony}
    به زبان قایل فهم برای چیپ ست است (بر مبنای
     \lr{api}
      های قابل درک برای بخش
       \lr{MT}
        ارتباط برقرار می کند).
\item \textbf{بخش سه
\lr{RIL Daemon}
:
}
  به نوعی وظیفه ترجمه میان
   \lr{RILJ}
    و
     \lr{Vendor RIL}
      را دارد.
     \end{enumerate}
  
   
            
\section*{سوال دوم}
نقش و جایگاه
 \lr{Qualcomm}
 ، 
 \lr{MediaTek}
 ،
 \lr{Exynos}
  و ... یا به عبارت بهتر تراشه سازهای
  \lr{MT}
   و
   \lr{Vendor}
   های گوشی های تلفن همراه نظیر
    \lr{Samsung}
    ،
     \lr{Nokia}
     ،
      \lr{Xiamoi}
       در این میان چیست؟

\subsection*{پاسخ}
نقش و جایگاه سازندگان تراشه مانند
 \lr{Qualcomm}
، 
\lr{MediaTek}
و
\lr{Exynos}
 و همچنین فروشندگان تلفن همراه مانند
  \lr{Samsung}
 ،
 \lr{Nokia}
 و
 \lr{Xiamoi}
  در رابطه با 
  \lr{RIL}
   را می‌توان به صورت زیر درک کرد:
\begin{enumerate}
	\item 
		\textbf{سازندگان تراشه:} 
		شرکت‌هایی مانند 
		\lr{Qualcomm}
		، 
		\lr{MediaTek}
		و
		\lr{Exynos}
		، تراشه‌های پردازنده موبایل را طراحی و تولید می‌کنند که شامل مودمی است که وظیفه اتصال 
		\lr{cellular}
		 یک دستگاه تلفن همراه را بر عهده دارد. این تراشه‌ها سخت‌افزار رادیویی را یکپارچه کرده و قابلیت‌های لازم برای اتصال به شبکه‌های تلفن همراه را فراهم می‌کنند. آنها همچنین درایورهای 
		 \lr{firmware}
		  و نرم‌افزار مورد نیاز برای تعامل مودم با
		   \lr{RIL}
		    را توسعه می‌دهند.
	\item 
	\textbf{پیاده سازی
	 \lr{RIL}
	 :}
	 \lr{RIL} 
	 معمولاً توسط سازنده دستگاه های تلفن همراه مانند 
	 \lr{Samsung}
	 ،
	 \lr{Nokia}
	 یا
	 \lr{Xiamoi}
	  بر اساس مشخصات و الزامات سخت افزار سازنده تراشه توسعه و پیاده سازی می شود. سازنده دستگاه تلفن همراه برای اطمینان از سازگاری و بهینه سازی 
\lr{RIL}
	   برای دستگاه های خاص خود، از نزدیک با سازنده تراشه همکاری می کند.
	   
	   \item 
	   \textbf{یکپارچه‌سازی
	    \lr{firmware}
	    :}
	    \lr{firmware }
	     و درایورهای نرم‌افزار سازنده تراشه برای مودم در پشته نرم‌افزار دستگاه تلفن همراه
	     (
	     \lr{mobile device's software stack}
	     )
	     ، از جمله سیستم‌عامل اندروید، ادغام شده‌اند. این یکپارچه سازی به 
	     \lr{RIL}
	      اجازه می دهد تا با مودم ارتباط برقرار کند و عملکردهای آن را کنترل کند.
\item 
\textbf{فروشندگان دستگاه های موبایل:} 
فروشندگان دستگاه های تلفن همراه، 
\lr{Samsung}
،
\lr{Nokia}
،
\lr{Xiamoi}
 و دیگران، از سخت افزار سازنده تراشه و 
 \lr{RIL}
  یکپارچه برای ساخت دستگاه های تلفن همراه خود استفاده می کنند. آنها پیاده سازی
   \lr{RIL}
    را برای مطابقت با ویژگی ها و الزامات خاص دستگاه خود سفارشی می کنند. آنها همچنین برای اطمینان از سازگاری، عملکرد و انطباق با استانداردهای شبکه تلفن همراه با سازنده تراشه همکاری نزدیکی دارند.
\end{enumerate}

به طور خلاصه، سازندگان تراشه مانند درایورهای سخت‌افزار، 
\lr{firmware}
 و نرم‌افزار را برای مودم ارائه می‌کنند، در حالی که فروشندگان دستگاه‌های تلفن همراه
  \lr{RIL}
   را پیاده‌سازی و سفارشی‌سازی می‌کنند تا ارتباط بین سیستم عامل اندروید را فعال کنند. سیستم و مودم در دستگاه های خود. این همکاری اتصال و عملکرد یکپارچه دستگاه های تلفن همراه را در شبکه های سلولی تضمین می کند.

\section*{سوال سوم}
سعی کنید اطلاعات
\lr{RIL}
 را توسط
 \lr{adb} 
 بگیرید و تحلیل کنید.
\begin{itemize}
	\item 
	 \lr{Data ON/OFF}
	 (خاموش و روشن کردن اینترنت)
\end{itemize}
\subsection*{پاسخ}
برای بررسی وصل بودن تلفن همراه در 
\lr{adb}
دستور لیست دستگاه های
\lr{adb}
را اجرا می کنیم. بدین منظور باید پس از اتصال دستگاه با 
\lr{USB}
به کامپیوتر، باید تنظیمات 
\lr{Developer mode}
را روی تلفن همراه فعال کرده و 
\lr{check box}
های 
\lr{stay awake}
و 
\lr{USB debugging}
را فعال کنیم. سپس دستورات زیر را در ترمینال اجرا کنیم:
\begin{latin}
	\lstinputlisting[language=Python]{Codes/DeverloperCommands.txt}
	\label{code:DeverloperCommands}
\end{latin}
پس از اجرای دستور آخر دستگاه های متصل به کامپیوتر مشاهده می شوند و کد آن ها را نیز می توان دید.
اکنون با استفاده از کد زیر 
\lr{log}
ها را در یک فایل 
\lr{text}
ذخیره می کنیم:
\begin{latin}
	\lstinputlisting[language=Java]{Codes/LogPrinter.java}
	\label{code:LogPrinter}
\end{latin}

در ابتدا به توضیح مفاهیم کلیدی مشاهده شده در 
\lr{log}
ها می پردازیم:
\begin{itemize}
\item
\textbf{\lr{APN} 
یا
\lr{Access Point Name}
:}
نام نقطه دسترسی که به طور خاص برای اتصال به یک نوع سرویس شبکه (مانند اینترنت) پیکربندی شده است.
\item
\textbf{\lr{APN Type}
:}
 نوع 
 \lr{APN}
 که می تواند "
 \lr{default}
 " برای اتصال به اینترنت عمومی یا "
 \lr{mms}
 " برای ارسال و دریافت پیام های چندرسانه ای باشد.
\item
\textbf{\lr{State}
:}
 وضعیت
  \lr{APN}
  ، که می تواند "
  \lr{DISCONNECTED}
  "، "
  \lr{CONNECTING}
  "، "
  \lr{CONNECTED}
  " یا "
  \lr{IDLE}
  " باشد.
\item
\textbf{\lr{Reason}
: }
دلیل تغییر وضعیت 
\lr{APN}
، که می تواند "
\lr{dataEnabled}
"، "
\lr{connected}
"، "
\lr{specificDisabled}
" یا "
\lr{other}
" باشد.

\end{itemize}
در تلفن همراه خود ابتدا 
\lr{Data}
در حالت خاموش قرار دارد. 
\lr{log}
برای این حالت به شکل زیر است:
\begin{latin}
	\lstinputlisting[language=Python]{Codes/state1.txt}
	\label{code:state1}
\end{latin}
\textbf{تجزیه و تحلیل:}
\begin{itemize}
	\item
	\textbf{رویداد
	\lr{APN}
	: }
	 نوع "
	 \lr{mms}
	 " به حالت "
	 \lr{DISCONNECTED}
	 " تغییر وضعیت می دهد.
	\item
	\textbf{دلیل
	\lr{dataEnabled}
	:}
	 نشان می دهد که اتصال داده به طور کلی توسط کاربر یا سیستم غیرفعال شده است.
	 \item
	\textbf{نتیجه:}
	 اینترنت برای ارسال و دریافت پیام های چندرسانه ای در این لحظه غیرفعال است.
\end{itemize}

\lr{Data}
را در تلفن همراه خود روشن می کنیم و پس از گذشت چند ثانیه آن را خاموش می کنیم. 
\lr{log}
های بدست آمده در هر مرحله به صورت زیر می باشند:
\begin{latin}
	\lstinputlisting[language=Python]{Codes/state2.txt}
	\label{code:state2}
\end{latin}
\textbf{تجزیه و تحلیل:}
\begin{itemize}
	\item
	\textbf{رویداد
		\lr{APN}
		: }
	نوع "
	\lr{default}
	" به حالت "
	\lr{CONNECTED}
	" تغییر وضعیت می دهد.
	\item
	\textbf{دلیل
		\lr{connected}
		:}
	نشان می دهد که اتصال به شبکه تلفن همراه برقرار شده و
	 \lr{APN}
	 "
	 \lr{default}
	" برای ارائه دسترسی به اینترنت انتخاب شده است.
	\item
	\textbf{نتیجه:}
	اینترنت برای مرور وب، استفاده از برنامه ها و سایر فعالیت های آنلاین در این لحظه فعال است.
\end{itemize}
\begin{latin}
	\lstinputlisting[language=Python]{Codes/state3.txt}
	\label{code:state3}
\end{latin}
\textbf{تجزیه و تحلیل:}
\begin{itemize}
	\item
	\textbf{رویداد
		\lr{APN}
		: }
	نوع "
	\lr{default}
	" به حالت "
	\lr{DISCONNECTED}
	" تغییر وضعیت می دهد.
	\item
	\textbf{دلیل
		\lr{specificDisabled}
		:}
	نشان می دهد که اتصال به 
	\lr{APN}
	 "
	 \lr{default}
	 "
	 به طور خاص توسط کاربر یا سیستم غیرفعال شده است، در حالی که اتصال داده به طور کلی هنوز فعال است.
	\item
	\textbf{نتیجه:}
	اینترنت برای مرور وب، استفاده از برنامه ها و سایر فعالیت های آنلاین در این لحظه غیرفعال است، اما امکان ارسال و دریافت پیام های چندرسانه ای (در صورت پیکربندی 
	\lr{APN}
	 "
	 \lr{mms}
	" جداگانه) وجود دارد.
\end{itemize}
\begin{latin}
	\lstinputlisting[language=Python]{Codes/state4.txt}
	\label{code:state4}
\end{latin}
\textbf{تجزیه و تحلیل:}
\begin{itemize}
	\item
	\textbf{رویداد
		\lr{APN}
		: }
	نوع "
	\lr{mms}
	" به حالت "
	\lr{DISCONNECTED}
	" تغییر وضعیت می دهد.
	\item
	\textbf{دلیل
		\lr{specificDisabled}
		:}
	مشابه رویداد 3، نشان می دهد که اتصال به 
	\lr{APN}
	 "
	 \lr{mms}
	 " به طور خاص غیرفعال شده است.
	\item
	\textbf{نتیجه:}
	ارسال و دریافت پیام های چندرسانه ای در این لحظه غیرفعال است.
\end{itemize}

\subsection*{تحلیل دقیق تر آخرین
\lr{log}
}
\begin{itemize}
	\item \lr{Source process: 4231}
	\item \lr{Target process: 4260}
	\item  سطح گزارش‌گیری
	 «\lr{D}»
	  برای پیام‌های اشکال‌زدایی (\lr{debugging}) است. این پیام‌ها حاوی جزئیات عمیق در مورد عملکرد سیستم هستند. از آن‌ها برای کمک به عیب‌یابی مشکلات استفاده می‌شود و معمولاً در شرایط عادی نمایش داده نمی‌شوند.
	  \item 
	  کامپوننت 
	 "\lr{DC-2}"
	   به مخفف "اتصال داده" 
	   (\lr{Data Connection}) 
	   است. این قطعه مسئول مدیریت ارتباط دستگاه با شبکه سلولی ({Cellular Network}\lr) است.
	  به عبارت دیگر، 
	  \lr{DC-2} 
	   نقش برقراری، حفظ و مدیریت اتصال گوشی شما به دکل‌های مخابراتی را بر عهده دارد. این اتصال به شما امکان می دهد تماس بگیرید، پیام ارسال کنید، از اینترنت همراه استفاده کنید و سایر کارهایی را که نیازمند برقراری ارتباط با شبکه سلولی هستند انجام دهید.
	  اگر با مشکلی در برقراری تماس، ارسال پیام یا استفاده از اینترنت همراه مواجه هستید، ممکن است خرابی در کامپوننت
	   \lr{DC-2}
	    علت آن باشد. البته این تنها یک احتمال است و عوامل دیگری نیز می‌توانند باعث ایجاد مشکل شوند.
	  \item 
	  پیام 
	  "\lr{Ignore setDetailed state}"
	   به این معنی است که درخواست برای تنظیم وضعیت "جزئیات" اتصال داده
	    \lr{MMS}
	    نادیده گرفته شده است. برای درک بهتر این پیام، بیایید اجزای آن را بررسی کنیم:
	 \begin{itemize}
	 	\item \textbf{\lr{Ignore}
	 	یا نادیده گرفتن:}  این کلمه نشان می‌دهد که اقدامی انجام نشده است.
	 	\item 
	\textbf{\lr{setDetailed state}
	یا تنظیم وضعیت جزئیات:} این عبارت اشاره به تلاش برای تغییر وضعیت اتصال داده 
	\lr{MMS} 
	به حالت "جزئیات" دارد.
	\item
\textbf{\lr{MMS}
یا پیام چندرسانه‌ای:} این نوع پیام، علاوه بر متن، می‌تواند شامل تصاویر، صدا و ویدیو نیز باشد.
	 \end{itemize}
	  بنابراین، کل پیام به این معنی است که سیستمی که این پیام را ثبت کرده، تلاشی برای تنظیم اطلاعات دقیق تر در مورد اتصال داده
	   \lr{MMS}
	    را نادیده گرفته است.
	  چند دلیل احتمالی برای این نادیده گرفتن وجود دارد:
	  \begin{itemize}
	  	 \item 
	  	 \textbf{عدم پشتیبانی:} ممکن است دستگاه یا نرم افزار به طور کلی از تنظیم وضعیت "جزئیات" برای اتصال داده 
	  	 \lr{MMS}
	  	  پشتیبانی نکند.
	  	\item
	  	\textbf{خطا:}
	  	 ممکن است در حین تلاش برای تنظیم وضعیت، خطایی رخ داده باشد که باعث نادیده گرفتن درخواست شده است.
	  	\item
	\textbf{تنظیمات غیرمجاز:} تنظیمات دقیق‌تر ممکن است به دلایل امنیتی یا پیکربندی خاص دستگاه، مجاز نباشد.
	  \end{itemize}
	 \item 
	 دلیل نادیده گرفتن درخواست، 
	 "\lr{dataEnabled}" 
	 (دیتا روشن) است، یعنی دیتای دستگاه از قبل روشن بوده است. این موضوع نشان می‌دهد که از آنجایی که دیتا به طور کلی روی دستگاه فعال است، نیازی به تغییر وضعیت جزئیات اتصال داده 
	 \lr{MMS}
	  نیست.
	 به عبارت دیگر، تنظیم وضعیت "جزئیات" برای اتصال داده 
	 \lr{MMS} 
	 احتمالا برای مدیریت دقیق‌تر مصرف اینترنت همراه در نظر گرفته شده است. اما از آنجایی که دیتای کل دستگاه روشن است، نیازی به اعمال این مدیریت دقیق‌تر روی
	  \lr{MMS}
	  وجود ندارد، چرا که هر گونه ترافیک اینترنت همراه، چه از طریق
	   \lr{MMS}
	   و چه از طریق برنامه‌های دیگر، از همان اتصال کلی دیتا عبور خواهد کرد.
	   
	   \item 
	   عبارت 
	   "\lr{not Required}"
	    (نیاز نیست) بر این نکته تأکید می‌کند که تنظیم وضعیت جزئیات در این مورد ضروری نیست.
	   این عبارت توضیح اضافی برای دلیل نادیده گرفتن درخواست است. با وجود اینکه 
	   "\lr{dataEnabled}" 
	   نشان می‌دهد دیتای کلی روشن است، عبارت 
	   "\lr{not Required}"
	    صراحتا بیان می‌کند که حتی اگر امکان تنظیم وضعیت جزئیات برای
	     \lr{MMS}
	     وجود داشت، در این شرایط خاص نیازی به انجام آن نبوده است.
	   
	 \item 
	 اطلاعات مربوط به متن 
	\lr{ APN (apnCnxt)} 
	 جزئیات بیشتری در مورد اتصال شبکه‌ای که برای
	  \lr{MMS}
	   استفاده می‌شود ارائه می‌دهد. این اطلاعات شامل موارد زیر است:
	   \begin{itemize}
\item
\textbf{نوع
	 \lr{APN}
 (\lr{APN Type}): }
 \lr{APN}
  مخفف "نام نقطه دسترسی" 
  (\lr{Access Point Name}) 
  است که دروازه ای برای برقراری ارتباط با شبکه داده تلفن همراه است. نوع 
  \lr{APN}
   مشخص می‌کند که برای چه نوع ترافیکی (مانند
    \lr{MMS}
    ، اینترنت، و غیره) در نظر گرفته شده است.
    \item
\textbf{وضعیت فعلی اتصال
 (\lr{Current Connection State}):} 
این بخش نشان می‌دهد که آیا اتصال
 MMS
 در حال حاضر برقرار است، در حال برقراری است یا قطع شده است.
\item
\textbf{فهرست
 \lr{APN}
  های در انتظار
  (\lr{List of Waiting APNs}):}
   در صورتی که اتصال با
    \lr{APN}
     فعلی برقرار نشود، دستگاه ممکن است 
  \lr{APN}
   های دیگری را برای برقراری ارتباط امتحان کند. این فهرست،
    \lr{APN}
     های جایگزینی را که در صف انتظار برای برقراری اتصال هستند، نشان می دهد.
     \item
   \textbf{  تنظیمات
      \lr{APN} 
     فعال فعلی 
     (\lr{Currently Active APN Configuration}): }
     این بخش جزئیات مربوط به 
     \lr{APN} 
     خاصی را که در حال حاضر برای اتصال 
     \lr{MMS} 
     استفاده می شود، ارائه می دهد. این تنظیمات ممکن است شامل مواردی مانند آدرس سرور
      \lr{APN}
     ، نام کاربری و رمز عبور باشد.
	   \end{itemize}
	 
	 بنابراین، اطلاعات
	  \lr{apnCnxt}
	   به درک چگونگی پیکربندی دستگاه برای برقراری ارتباط
	    \lr{MMS}
	     با شبکه تلفن همراه کمک می کند. این اطلاعات می تواند در عیب یابی مشکلات
	      \lr{MMS}
	       مفید باشد.
	\item 
	پارامترهای
	 mDataEnabled=true
	  و
	   mDependencyMet=true
	    تأیید می‌کنند که:
	    \begin{itemize}
	    	\item
	    	\textbf{\lr{mDataEnabled=true:}}
	    	 دیتای دستگاه روشن است. این همانطور که قبلاً بحث کردیم، دلیل اصلی نادیده گرفته شدن درخواست برای تنظیم وضعیت جزئیات اتصال داده
	    	  \lr{MMS}
	    	   است.
	    	\item
	    	\textbf{\lr{mDependencyMet=true:}}
	    	 هر گونه وابستگی 
	    	 (\lr{dependency})
	    	  برای تنظیم وضعیت جزئیات برآورده شده است. این بخش کمی فنی‌تر است، اما به طور کلی به این معنی است که هر شرط یا الزام دیگری که برای تنظیم وضعیت جزئیات اتصال داده 
	    	  \lr{MMS}
	    	   لازم بوده، برآورده شده است.
	    \end{itemize}
	این پارامترها با هم نشان می‌دهند که از نظر سیستم، همه چیز برای برقراری اتصال 
	\lr{MMS}
	 آماده است. دیتا روشن است و هر الزام دیگری نیز برآورده شده است. با این حال، به دلیل اینکه دیتای کلی روشن است، تنظیم دقیق‌تر اتصال داده
	  \lr{MMS}
	   در این شرایط خاص ضروری نیست.
	\item 
	تکرار عبارت
	 \lr{apnType = mms}
	 تأکید می‌کند که این گزارش به طور خاص به اتصال داده
	  \lr{MMS}
	   مربوط است.
	دلیل اهمیت این تکرار این است که:
	\begin{itemize}
		\item
		همانطور که قبلاً اشاره شد، 
		\lr{APN}
		مخفف "نام نقطه دسترسی" است و انواع مختلفی از
		\lr{APN}
		برای اهداف مختلف وجود دارد. یک
		\lr{APN}
		برای اینترنت، یک
		\lr{APN}
		دیگر برای
		\lr{MMS}
		و غیره وجود دارد.
		
		\item 
		ذکر مجدد
		 \lr{apnType = mms}
		 اطمینان می‌دهد که خواننده گزارش به طور واضح متوجه شود که اطلاعات ارائه شده در مورد اتصال داده‌ای است که به طور خاص برای ارسال و دریافت پیام‌های چندرسانه‌ای 
		 (\lr{MMS})
		  استفاده می‌شود.
		این تکرار به جلوگیری از هرگونه ابهام و اطمینان از درک صحیح ماهیت گزارش کمک می‌کند.
	\end{itemize}
	
\end{itemize}


حالت های مختلف در جدول
\ref{tab:ril-event-log}
به طور خلاصه ذکر شده اند.
\begin{table}[htbp]
	\centering
	\begin{tabular}{|c|c|c|}
		\hline
		\textbf{زمان} & \textbf{Reason} & \textbf{رویداد} \\
		\hline
		\lr{13:46:57.611}
		&
		\lr{dataEnabled}
		&
		اینترنت خاموش شد
		\\
		\hline
		\lr{13:46:57.752}
		&
		\lr{connected}
		&
		اینترنت روشن شد
		 \\
		\hline
		\lr{13:47:03.048}
		&
		\lr{specificDisabled}
		&
		اینترنت خاموش شد
		 \\
		\hline
	\end{tabular}
	\caption{\lr{RIL Event Log}}
	\label{tab:ril-event-log}
\end{table}


\textbf{نتیجه گیری:}
\begin{itemize}
	\item 
	بین
	\lr{13:46:57.752} 
	و
	\lr{13:47:03.048} 
	اینترنت برای مرور وب، استفاده از برنامه ها و سایر فعالیت های آنلاین فعال بود.
	\item 
	در 
	\lr{13:47:03.048}
	،
	 هر دو
	  \lr{APN} 
	 "
	 \lr{default}
	 " و "
	 \lr{mms}
	 " به طور خاص غیرفعال شدند، به این معنی که هم اینترنت و هم ارسال و دریافت پیام های چندرسانه ای در آن لحظه غیرفعال بودند.
\end{itemize}

\section* {مراجع}
\begin{itemize}
	\item 
	\href{https://www.oreilly.com/library/view/android-hackers-handbook/9781118608647/9781118922255c11.xhtml} {RIL}
	\item 
	\href{https://en.wikipedia.org/wiki/Radio_Interface_Layer} {RIL}
	
	\item 
	\href{https://en.wikipedia.org/wiki/Radio_Interface_Layer} {سوال اول}
	
	\item 
	\href{https://stackoverflow.com/questions/21170392/my-android-device-does-not-appear-in-the-list-of-adb-devices} {نمایش لیست دستگاه های 
	\lr{adb}
	}
	
	\item 
	\href{https://en-gb.support.motorola.com/app/answers/detail/a_id/159678/~/developer-options} {روشن کردن 
	\lr{Developer Mode}
	}
	
\end{itemize}
\end{document}
